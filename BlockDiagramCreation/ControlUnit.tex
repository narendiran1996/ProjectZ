\documentclass{standalone}

\usepackage[siunitx, RPvoltages]{circuitikz}
\usepackage{tikz}
\usepackage{xcolor}

\begin{document}

\begin{circuitikz}
    \ctikzset{multipoles/thickness=3}
    \ctikzset{multipoles/dipchip/width=1}
    \draw (0,0) node[dipchip,
        blue,
        rounded corners,
        scale =1,
        text width=1cm,
        num pins=14, hide numbers, no topmark,
        external pins width=0](C){\small Conrol\\ Unit};

    \draw (C.n) ++(-0.1,0) -- ++(0.1,-0.1) -- ++(0.1,0.1);
    \draw (C.n) -- ++(0,0.25) coordinate(clkPin) node[above,pos=1, font=\tiny]{\textcolor{blue}{CLK}};



    \draw (C.bpin 1) -- ++(-0.5,0) coordinate(CUpin1);
    \draw (C.bpin 2) -- ++(-0.5,0) coordinate(CUpin2);
    \draw (C.bpin 3) -- ++(-0.5,0) coordinate(CUpin3);
    \draw (C.bpin 7) -- ++(-0.5,0) coordinate(CUpin4);

    \draw (C.bpin 8) -- ++(0.5,0) coordinate(CUpin11);
    \draw (C.bpin 9) -- ++(0.5,0) coordinate(CUpin10);
    \draw (C.bpin 10) -- ++(0.5,0) coordinate(CUpin9);
    \draw (C.bpin 11) -- ++(0.5,0) coordinate(CUpin8);
    \draw (C.bpin 12) -- ++(0.5,0) coordinate(CUpin7);
    \draw (C.bpin 13) -- ++(0.5,0) coordinate(CUpin6);
    \draw (C.bpin 14) -- ++(0.5,0) coordinate(CUpin5);



\end{circuitikz}

\end{document}