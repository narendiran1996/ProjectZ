\documentclass{standalone}

\usepackage[siunitx, RPvoltages]{circuitikz}
\usepackage{tikz}

\begin{document}

\begin{circuitikz}
    \ctikzset{multipoles/thickness=3}
    \ctikzset{multipoles/dipchip/width=2}
    \draw (0,0) node[dipchip,
        scale =1.5,
        num pins=12, hide numbers, no topmark,
        external pins width=0](C){\small Register File};

    \node [right, font=\tiny] at (C.bpin 1) {RD\_addr1};
    \node [right, font=\tiny] at (C.bpin 2) {RD\_addr2};
    \node [right, font=\tiny] at (C.bpin 3) {WR\_addr3};
    \draw (C.bpin 1) -- ++(-0.5,0) coordinate(extpin1);
    \draw (C.bpin 2) -- ++(-0.5,0) coordinate(extpin2);
    \draw (C.bpin 3) -- ++(-0.5,0) coordinate(extpin3);
    \draw (C.bpin 5) -- ++(-0.5,0) coordinate(extpin4);
    \draw (C.bpin 6) -- ++(-0.5,0) coordinate(extpin5);
    
    \draw (C.bpin 5) ++(0,0.1) -- ++(0.1,-0.1)
    node[right, font=\tiny]{CLK} -- ++(-0.1,-0.1);
    \node [right, font=\tiny] at (C.bpin 6) {WR\_data};

    \draw (C.n) -- ++(0,0.5) node[below, pos=0] {\tiny{EN}};

    \node [left, font=\tiny] at (C.bpin 12) {RD\_data1};
    \node [left, font=\tiny] at (C.bpin 11) {RD\_data2};

    \draw (C.bpin 12) -- ++(0.5,0) coordinate(extpin6);
    \draw (C.bpin 11) -- ++(0.5,0) coordinate(extpin7);
\end{circuitikz}

\end{document}